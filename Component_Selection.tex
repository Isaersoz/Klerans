\section{Component Selection}
Other components of the isolated SEPIC converter is selected as given below. Considering the requirements of the topology and safety measures, we have selected potential candidates for the components.

\subsection{Controller Selection}
For the duty cycle calculation against the variation in input and output sides, and therefore gathering the current and voltage measurements from the sensors, we have decided our controller to be a digital one and we have selected the STM32-Nucleo-F334R8 board. Related information about the controller is given in Table \ref{tab:controller}.

\begin{table}[H]
    \centering
    \caption{STM32-Nucleo-F334R8 board information.}
    \label{tab:controller}
    \begin{tabular}{|c|c|}
        \hline
        Supply voltage:         & 5 V / 7-12 V   \\
        \hline
        Max. CPU frequency:     & 72 MHz        \\
        \hline
        Dimensions in mm:       & 70x82.5       \\
        \hline
        Cost:                   & 330 TL        \\
        \hline
    \end{tabular}
\end{table}

\subsection{Sensors}
In order to adjust the duty cycle to compensate the changes in input and outputs of the converter, we should monitor the input and output voltages continuously. For this purpose we have selected an isolated Delta-Sigma modulator product for our voltage measurements. Its specifications are given in below Table \ref{tab:AMC}. This product gives separates input from the output via capacitive double isolation barrier. The device can also be used for current measurements. Additional shunt resistances are not mentioned in this section.

\begin{table}[H]
    \centering
    \caption{Isolated Delta-Sigma Modulator Specifications.\cite{web:AMC}}
    \label{tab:AMC}
    \begin{tabular}{|c|c|}
        \hline
        Part no:                & AMC1306M25DWVR    \\
        \hline
        Sampling rate:          & 78k/s             \\
        \hline
        Isolation strength:     & 7000 V            \\
        \hline
        Cost:                   & 143.5 TL          \\
        \hline
    \end{tabular}
\end{table}


\subsection{Switching Components}
Looking at the topology as given in Figure \ref{fig:iso_sepic}, we have two switching components which are the MOSFET at the primary side and the diode at the secondary side of the transformer. Looking at the simulation results in Section \ref{comp_simulation}, we have selected our switching components with the addition of safety factors. For the MOSFETs, we decided to parallel two MOSFET to decrease to losses due to $R_{ds,on}$.

\begin{table}[H]
    \centering
    \caption{MOSFET Specifications.\cite{web:mosfet}, will be paralleled.}
    \label{tab:mosfet}
    \begin{tabular}{|c|c|}
        \hline
        Part no:            &  RSS070N05FRA     \\
        \hline
        $V_{ds}$            &  45 V             \\
        \hline
        $I_d$               &  7 A              \\
        \hline
        $R_{ds,on}$         & 25 $m\Omega$      \\
        \hline
        Total gate charge:  & 12 nC             \\
        \hline
        Cost:               & 15.2 TL           \\
        \hline
    \end{tabular} 
\end{table}

To drive these MOSFETS properly, we have selected an isolated gate driver with specifications below in Table \ref{tab:gate_driver}.

\begin{table}[H]
    \centering
    \caption{Gate Driver Product Specifications.\cite{web:gate_driver}}
    \label{tab:gate_driver}
    \begin{tabular}{|c|c|}
         \hline
         Part no:               & UCC23313BDWYR     \\
         \hline
         Current output peak:   & 5.3 A             \\
         \hline
         Output supply:         & 14-33 V           \\
         \hline
         Cost:                  & 33.1 TL           \\
         \hline
    \end{tabular}
\end{table}

For the diode, we have selected the following Schottky diode product for our design as given below. Again we are not limited to this product, we can select another Schottky diode with similar ratings for our design if this one is not available during the implementation.

\begin{table}[H]
    \centering
    \caption{Schottky Diode Specifications.\cite{web:diode}}
    \label{tab:diode}
    \begin{tabular}{|c|c|}
        \hline
        Part no:            & VSSAF512          \\
        \hline
        $V_r$               & 12 V              \\
        \hline
        $I_f$               & 5 A               \\
        \hline
        $V_{f,max}$         & 0.88 V            \\
        \hline
        Cost:               & 7.6 TL            \\
        \hline
    \end{tabular}
\end{table}


\subsection{Inductor Core Selection}
For the inductor at the input side, in order not to saturate the core and obtain the desired inductance value, we have selected a Kool Mu core with its specifications below. Again we will be looking for alternative cores for our inductor design, considering our available space and limitations.

\begin{table}[H]
    \centering
    \caption{Inductor Toroid core specifications.\cite{web:tor_core}}
    \label{tab:toroid_core}
    \begin{tabular}{|c|c|}
        \hline
        Part no:            & 0077548A7         \\
        \hline
        Core material:      & Kool Mu           \\
        \hline
        Permeability:       & 125               \\
        \hline
        Inductance Ratio:   & 127 $nH/T^2$      \\
        \hline
        Dimensions in mm:   & 33x33x11          \\
        \hline
        Cost:               & 52 TL             \\
        \hline
    \end{tabular}
\end{table}

If we select and wind this core with $N=16$ turns to get a $32.6\mu H$ inductor with AWG26 wire with the same paralleling configuration for the primary side, we get the ESR of the inductance as given below. Using the core dimensions, an estimate length for each winding is calculated as 48 mm.

\begin{equation}
    ESR_{L}=\frac{48 mm\times16\times133 m\Omega/m\times10^{-3}m/mm}{12}=8.6 m\Omega
\end{equation}

\subsection{Capacitor Selection}
Selecting the input capacitor, we desired a large enough capacitor with enough current rating. We have decided on paralleling three of the Aluminum capacitor with given parameters in Table \ref{tab:input_cap}. Since we are paralleling the capacitors, if the component will not be available, we can select smaller capacitance with similar ratings but this will result in higher volumes of the converter.

\begin{table}[H]
    \centering
    \caption{Input Capacitor Specification.\cite{web:input_cap}}
    \label{tab:input_cap}
    \begin{tabular}{|c|c|}
        \hline
        Part no:        & KR3-035V332MJ250  \\
        \hline
        Capacitance:    & 3300 $\mu F$      \\
        \hline
        Rated voltage:  & 35 $V$             \\
        \hline
        Ripple current: & 1960 $mA$         \\
        \hline
        Dimensions in mm & 16x16x25          \\
        \hline
        ESR:            & not specified     \\
        \hline
        Cost:           & 11 TL             \\
        \hline
    \end{tabular}
\end{table}

For the output capacitor, since the output voltage ripple is desired to be very low, we agreed on putting a 1 mF capacitor to the output side. Parameters of the candidate capacitor is given below. We are not limited to using this capacitor, we can select another one with similar ratings in case of unavailability.

\begin{table}[H]
    \centering
    \caption{Output Capacitor Specification.\cite{web:output_cap}}
    \label{tab:output_Cap}
    \begin{tabular}{|c|c|}
        \hline
        Part no:        & KLH-100V102MK400  \\
        \hline
        Capacitance:    & 1000 $\mu F$      \\
        \hline
        Rated Voltage:  & 100 V             \\
        \hline
        Ripple current: & 2500 $mA$         \\
        \hline
        ESR:            & 76 $m\Omega$      \\
        \hline
        Dimensions(mm)  & 18x18x40          \\
        \hline
        Cost:           & 23.3 TL           \\
        \hline
    \end{tabular}
\end{table}